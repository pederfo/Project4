%% file: template.tex = LaTeX template for article-like report 
%% init: sometime 1993
%% last: Feb  8 2015  Rob Rutten  Deil
%% site: http://www.staff.science.uu.nl/~rutte101/rrweb/rjr-edu/manuals/student-report/

%% First read ``latex-bibtex-simple-manual.txt'' at
%% http://www.staff.science.uu.nl/~rutte101/Report_recipe.html

%% Start your report production by copying this file into your XXXX.tex.
%% Small changes to the header part will make it an A&A or ApJ manuscript.

%%%%%%%%%%%%%%%%%%%%%%%%%%%%%%%%%%%%%%%%%%%%%%%%%%%%%%%%%%%%%%%%%%%%%%%%%%%%
\documentclass{aa}   %% Astronomy & Astrophysics style class
\usepackage{graphicx,natbib,url,twoopt}
\usepackage[varg]{txfonts}           %% A&A font choice
\usepackage{hyperref}                %% for pdflatex
%%\usepackage[breaklinks]{hyperref}  %% for latex+dvips
%%\usepackage{breakurl}              %% for latex+dvips
\usepackage{pdfcomment}              %% for popup acronym meanings
\usepackage{acronym}                 %% for popup acronym meanings

\hypersetup{
  colorlinks=true,   %% links colored instead of frames
  urlcolor=blue,     %% external hyperlinks
  linkcolor=red,     %% internal latex links (eg Fig)
}

\bibpunct{(}{)}{;}{a}{}{,}    %% natbib cite format used by A&A and ApJ

\pagestyle{plain}   %% undo the fancy A&A pagestyle 

%% Add commands to add a note or link to a reference
\makeatletter
\newcommand{\bibnote}[2]{\@namedef{#1note}{#2}}
\newcommand{\biblink}[2]{\@namedef{#1link}{#2}}
\makeatother

%% Commands to make citations ADS clickers and to add such also to refs
%% May 2014: they give error stops ("Illegal parameter number ..."}
%%   for plain latex with TeX Live 2013; the ad-hoc fixes added below let
%%   latex continue instead of stop within these commands.
%%   Please let me know if you know a better fix!
%%   No such problem when using pdflatex.
\makeatletter
 \newcommandtwoopt{\citeads}[3][][]{%
   \nonstopmode%              %% fix to not stop at error message in latex
   \href{http://adsabs.harvard.edu/abs/#3}%
        {\def\hyper@linkstart##1##2{}%
         \let\hyper@linkend\@empty\citealp[#1][#2]{#3}}%   %% Rutten, 2000
   \biblink{#3}{\href{http://adsabs.harvard.edu/abs/#3}{ADS}}%
   \errorstopmode}            %% fix to resume stopping at error messages 
 \newcommandtwoopt{\citepads}[3][][]{%
   \nonstopmode%              %% fix to not stop at error message in latex
   \href{http://adsabs.harvard.edu/abs/#3}%
        {\def\hyper@linkstart##1##2{}%
         \let\hyper@linkend\@empty\citep[#1][#2]{#3}}%     %% (Rutten 2000)
   \biblink{#3}{\href{http://adsabs.harvard.edu/abs/#3}{ADS}}%
   \errorstopmode}            %% fix to resume stopping at error messages
 \newcommandtwoopt{\citetads}[3][][]{%
   \nonstopmode%              %% fix to not stop at error message in latex
   \href{http://adsabs.harvard.edu/abs/#3}%
        {\def\hyper@linkstart##1##2{}%
         \let\hyper@linkend\@empty\citet[#1][#2]{#3}}%     %% Rutten (2000)
   \biblink{#3}{\href{http://adsabs.harvard.edu/abs/#3}{ADS}}%
   \errorstopmode}            %% fix to resume stopping at error messages 
 \newcommandtwoopt{\citeyearads}[3][][]{%
   \nonstopmode%              %% fix to not stop at error message in latex
   \href{http://adsabs.harvard.edu/abs/#3}%
        {\def\hyper@linkstart##1##2{}%
         \let\hyper@linkend\@empty\citeyear[#1][#2]{#3}}%  %% 2000
   \biblink{#3}{\href{http://adsabs.harvard.edu/abs/#3}{ADS}}%
   \errorstopmode}            %% fix to resume stopping at error messages 
\makeatother

%% Acronyms
\newacro{ADS}{Astrophysics Data System}
\newacro{NLTE}{non-local thermodynamic equilibrium}
\newacro{NASA}{National Aeronautics and Space Administration}

%% Add popups with meaning to acronyms 
%% NB: only show up in Adobe Reader and do not work with \input or \include
\gdef\acp#1{%
  \pdfmarkupcomment[markup=Underline,color={1 1 1},author={{#1}},opacity=0]%
  {{#1}}{{\acl{#1}}}}

%% Spectral species
\def\MgI{\ion{Mg}{I}}          %% A&A; for aastex use \def\MgI{\ion{Mg}{1}} 
\def\MgII{\ion{Mg}{II}}        %% A&A; for aastex use \def\MgII{\ion{Mg}{2}} 

%% Hyphenation
\hyphenation{Schrij-ver}       %% Dutch ij is a single character

%%%%%%%%%%%%%%%%%%%%%%%%%%%%%%%%%%%%%%%%%%%%%%%%%%%%%%%%%%%%%%%%%%%%%%%%%%%%
\begin{document}  

%% simple header.  Change into A&A or ApJ commands for those journals

\twocolumn[{%
\vspace*{4ex}
\begin{center}
  {\Large \bf FYS4150 Project 4: Numerical integration}\\[4ex]
  {\large \bf Peder Forfang, Andreas Ellewsen}\\[4ex]
  %{\large \bf Andreas Ellewsen$^{1}$}\\[4ex]
  %\begin{minipage}[t]{15cm}
  %      $^1$ Institute of theoretical astrophysics\\

%  {\bf Abstract.} We learned how to write nice reports \ldots 

  %\vspace*{2ex}
  %\end{minipage}
\end{center}
}]
%%%%%%%%%%%%%%%%%%%%%%%%%%%%%%%%%%%%%%%%%%%%%%%%%%%%%%%%%%%%%%%%%%%%%%%%%%%%
\section{Introduction}\label{sec:introduction}
%%%%%%%%%%%%%%%%%%%%%%%%%%%%%%%%%%%%%%%%%%%%%%%%%%%%%%%%%%%%%%%%%%%%%%%%%%%%
In this project we study the Ising model in two dimensions, without an external magnetic field. 
In its simplest form the energy is expressed  as 
\begin{equation}
 E = -J\sum_{<kl>}^N s_ks_l
\end{equation}
with $s_k = \pm 1$, $N$ is the total number of spins and $J$ is a coupling constant expressing the strength of the interaction between neighbouring spins. The symbol $<kl>$ indicates that we sum over nearest neighbours only. We will assume that we have a ferromagnetic ordering, viz $j > 0$. We also assume periodic boundary conditions. This project will be solved using the Metropolis algorithm.
%%%%%%%%%%%%%%%%%%%%%%%%%%%%%%%%%%%%%%%%%%%%%%%%%%%%%%%%%%%%%%%%%%%%%%%%%%%%
\section{Analytical solution}\label{sec:analytical}
%%%%%%%%%%%%%%%%%%%%%%%%%%%%%%%%%%%%%%%%%%%%%%%%%%%%%%%%%%%%%%%%%%%%%%%%%%%%
To start with we assume that we only have two spins in each direction, such that $L = 2$, giving us a total of 4 spins.
This gives a partition function 
\begin{equation}
 Z = 12 + 4cosh(8J\beta)
\end{equation}
where $\beta = 1/kT$.
An expectation value for energy
\begin{equation}
 <E> = -\frac{32Jsinh(8J\beta)}{Z}\frac{1}{L^2}
\end{equation}
A variance for the energy
\begin{equation}
 \sigma_E^2 = \frac{2^{10}(1+3cosh(8J\beta))}{Z^2}\frac{1}{L^2}
\end{equation}
Expectation value for the magnetization
\begin{equation}
 <M> = 0.
\end{equation}
Expectation value for the absolute value of the magnetization
\begin{equation}
 <|M|> = \frac{8(e^{8J\beta}+2)}{Z}\frac{1}{L^2}
\end{equation}
Variance of the magnetization
\begin{equation}
 \sigma_M^2 = \frac{2^5(e^{8J\beta}+1)}{Z}\frac{1}{L^2} - <|M|>^2
\end{equation}
XXXXXXXXXXXXX THIS CALCULATION IS WRONG XXXXXXXXXXXXX 
which gives us equations for specific heat capacity
\begin{equation}
C_V = \frac{\sigma_E^2}{kT^2}
\end{equation}
and magnetic susceptibility
\begin{equation}
 \chi = \frac{\sigma_M^2}{kT}
\end{equation}

All these values will prove very useful for verifying that the program simulating larger systems is running correctly.

%%%%%%%%%%%%%%%%%%%%%%%%%%%%%%%%%%%%%%%%%%%%%%%%%%%%%%%%%%%%%%%%%%%%%%%%%%%%
\section{Simulating the L=2 system}\label{sec:simulate_analytic}
%%%%%%%%%%%%%%%%%%%%%%%%%%%%%%%%%%%%%%%%%%%%%%%%%%%%%%%%%%%%%%%%%%%%%%%%%%%%
It is now time to simulated the system using the Metropolis algorithm. The program we write computes the mean energy $<E>$, mean magnetization $<|M|>$, the specific heat $C_V$ and the susceptibility $\chi$ as functions of T. To start with we want to check that our program works correclty by checking our results against the ones for the L=2 system. We do this for temperature $T = 1$ (in units $kT/J$).

By looking at the expectation values versus the number of Monte Carlo cycles we can estimate how many cycles are needed before we get a value close to the analytical one. Depending on what one defines as a ``good'' approximation we get either a very good match from the beginning, or a good match after about 300,000 cycles. Note that both expectation values for energy and magnetization vary only by one thousandth. See figures \ref{expecE2} and \ref{expecM2}.

\begin{figure}
 \includegraphics[width=.49\textwidth]{expecE2.png}
 \caption{Plot shows he computed $<E>$ versus the number of Monte Carlo cycles.In this case $T=1$ (in units of $kT/J$), and L=2}
\label{expecE2}
\end{figure}

\begin{figure}
 \includegraphics[width=.49\textwidth]{expecM2.png}
 \caption{Plot shows he computed $<|M|>$ versus the number of Monte Carlo cycles. In this case $T=1$ (in units of $kT/J$), and L=2}
\label{expecM2}
\end{figure}

\begin{figure}
 \includegraphics[width=.49\textwidth]{expecEM2.png}
 \caption{The figure shows the expectation values for energy and the absolute value of the magnetization. Note that with such a low temperature the values stabilize from the start, and the variations are too small to see at this scale.}
\label{expecEM2}
\end{figure}

Comparing the output from the program with the values we calculated earlier gives the following table
\begin{table}
 \begin{tabular}{|c|c|c|}
  \hline
  &Analytical &Numerical \\
  \hline
  $<E>$ &-1.996 & -1.996\\
  \hline
  $<|M|>$& 0.999& 0.999\\
  \hline
  $C_V$ & 0.032& 0.032\\
  \hline
  $\chi$ & 2.996& 0.004\\
  \hline
 \end{tabular}
\caption{Table of values from the analytial calculations for a system with L=2 and T = 1 in units of (kT/J). Note the precision of the numerical result.}
\end{table}

%%%%%%%%%%%%%%%%%%%%%%%%%%%%%%%%%%%%%%%%%%%%%%%%%%%%%%%%%%%%%%%%%%%%%%%%%%%%
\section{Larger systems}\label{sec:Monte Carlo import}
%%%%%%%%%%%%%%%%%%%%%%%%%%%%%%%%%%%%%%%%%%%%%%%%%%%%%%%%%%%%%%%%%%%%%%%%%%%%
Since we have now verified that our program runs correctly we want to increase the size of the lattice. We start by increasing the number of spins to $L = 20$ in each direction.
We did not study how many cycles were needed to reach the most likely state in the last section. This is something we should study so that we know how much time is needed to reach an equilibrium state, and so how long we have to wait before we start computing our expectation values. We start by making an estimate by plotting the the expectations values as function of the number of Monte Carlo cycles. This is done for $T = 1$

XXXXXXXXXXXXX INSERT PLOT FOR VARIOUS EXPECTATION VALUES AS FUNCTION OF NUMBER OF CYCLES XXXXXXXXXXXXX

We see that XXXX WHAT DO WE SEE XXXX. 

%%%%%%%%%%%%%%%%%%%%%%%%%%%%%%%%%%%%%%%%%%%%%%%%%%%%%%%%%%%%%%%%%%%%%%%%%%%%
\section{Conclusions} \label{sec:conclusions}
%%%%%%%%%%%%%%%%%%%%%%%%%%%%%%%%%%%%%%%%%%%%%%%%%%%%%%%%%%%%%%%%%%%%%%%%%%%%

%%%%%%%%%%%%%%%%%%%%%%%%%%%%%%%%%%%%%%%%%%%%%%%%%%%%%%%%%%%%%%%%%%%%%%%%%%%%
\section{Source code}\label{sec:source}
%%%%%%%%%%%%%%%%%%%%%%%%%%%%%%%%%%%%%%%%%%%%%%%%%%%%%%%%%%%%%%%%%%%%%%%%%%%%
The source code for this document, the c++ project, and the python program for plotting can be found at \url{https://github.com/pederfo/Project4}.

%%%%%%%%%%%%%%%%%%%%%%%%%%%%%%%%%%%%%%%%%%%%%%%%%%%%%%%%%%%%%%%%%%%%%%%%%%%%
%\begin{acknowledgements}
%\end{acknowledgements}

%%%%%%%%%%%%%%%%%%%%%%%%%%%%%%%%%%%%%%%%%%%%%%%%%%%%%%%%%%%%%%%%%%%%%%%%%%%%
%% references
\section{References}
Computational Physics, Lecture notes Fall 2015, Morten Hjorth-Jensen

%\bibliographystyle{aa-note} %% aa.bst but adding links and notes to references
%\raggedright              %% only for adsaa with dvips, not for pdflatex
%\bibliography{XXX}          %% XXX.bib = your Bibtex entries copied from ADS

\end{document}